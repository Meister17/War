\documentclass[a4paper,12pt]{article}
\usepackage[T2A]{fontenc}
\usepackage[russian]{babel}
\usepackage[utf8]{inputenc}
\usepackage{indentfirst}
\date{\today}

\begin{document}
	\begin{titlepage}
		\begin{center}
			\large Московский Государственный Университет им. М.В.Ломоносова\\[4,5cm]
			\huge Конспект лекций \\[0,6cm]
			\large по курсу "Тактическая подготовка"\\[3,7cm]
				\begin{minipage}{0,5\textwidth}
					\begin{flushleft}
						\emph{Автор:} Нокель Михаил\\
						\emph{Группа:} 420\\
						\emph{Факультет:} ВМК\\
						\emph{Преподаватель:} Крылов Константин Эдуардович\\
					\end{flushleft}					
				\end{minipage}	
			\vfill
			{\large \today}
			{\large \LaTeX}			
		\end{center}
	\thispagestyle{empty}
	\end{titlepage}
	\newpage
	\tableofcontents
	\newpage
	\section{\bf Система управления в соединении ПВО}
	Основные принципы управления:
	\begin{enumerate}
		\item Принцип единоначалия
		\item Централизация управления с предоставлением подчинённым инициативы в определении
		способов решения поставленных им задач
		\item Твёрдость и настойчивость проведения принятых решений
		\item Оперативность и гибкость при реагировании на изменения обстановки
		\item Личная ответственность за принимаемые решения, а также за результаты выполнения
		поставленных боевых задач подчинёнными подразделениями
		\item Высокая организованность и творчество в работе командиров и штабов
	\end{enumerate}
	Управление включает в себя:
	\begin{enumerate}
		\item непрерывное добывание, сбор, обработку, изучение, обобщение, анализ, оценку и 
		отображение данных складывающейся обстановки с учётом прогноза её развития при
		подготовке боевых действий, в ходе их ведения и после выполнения боевой задачи
		\item принятие решения, доведение задач до подчинённых частей и подразделений
		\item планирование боевых действий и других видов боевой деятельности
		\item организацию и поддержание взаимодействия
		\item организацию и проведение мероприятий по видам обеспечения
		\item руководство подготовкой нижестоящих штабов, соединений воинских частей и 
		подразделений к боевым действиям
		\item организацию и осуществление контроля и помощи нижестоящим штабам, соединениям, 
		частям и подразделениям
		\item непосредственное руководство подчинёнными частями и подразделениями
		\item поддержание высокого морально-психологического состояния подчинённых соединений, 
		воинских частей и подразделений
	\end{enumerate}
	Управление должно быть:
	\begin{enumerate}
		\item устойчивым
		\item непрерывным
		\item оперативным
		\item скрытным
	\end{enumerate} 
	\underline{Основой управления является решение командира!}
	
	Для обеспечения эффективного и качественного решения основных задач управления подчинёнными
частями и подразделениями требуется:
	\begin{enumerate}
		\item высокий уровень подготовки командиров и штабов
		\item глубокое понимание ими характера современных войн, а также способов ведения 
		боевых действий
		\item твёрдое знание боевых возможностей и основ применения вооружения и военной 
		техники, стоящих на вооружении в подчинённых частях и подразделениях
		\item знание боевых возможностей противника
		\item умение правильно и всесторонне оценивать обстановку и принимать обоснованные решения
		\item предвидение, предусмотрительность, разумная инициатива и высокая оперативность
		в работе при принятии решений, постановки или уточнения задач подчинённым, при 
		планировании боевых действий
		\item умелое применение средств управления войсками и оружием
		\item надёжная защита пунктов управления
	\end{enumerate}
	В корпусе (или дивизии ПВО) создаётся система управления, представляющая собой совокупность
функционально связанных:
	\begin{enumerate}
		\item органов управления
		\item пунктов управления
		\item средств управления
	\end{enumerate}
	Она должна обладать следующими чертами:
	\begin{enumerate}
		\item высокой живучестью
		\item помехозащищённостью
		\item надёжностью
		\item способностью обеспечивать возможность как централизованного, так и 
		децентрализованного управления подчинёнными частями и подразделениями
	\end{enumerate}
	\newpage
	\centerline{04 марта 2010г.}
	К органам управления относят:
	\begin{enumerate}
		\item командование
		\item штаб
		\item отделы, службы и другие постоянные и временно создаваемые управленческие 
		организации
	\end{enumerate}
	В корпусе или дивизии ПВО могут быть следующие командные пункты:
	\begin{enumerate}
		\item основной командный пункт (тот пункт, с которого командир может управлять 
		подчинёнными как в мирное, так и в военное время)
		\item запасной командный пункт (развертывается в целях повышения устойчивости и 
		непрерывности управления). Может быть как в стационарном виде, так и в подвижном. 
		Назначается на базе командного пункта подчинённой части. 
		\item тыловой пункт управления, который предназначен для управления тылом, для 
		руководства по организации тылового обеспечения частей и подразделений. Как правило 
		располагается в основном командном пункте. Возглавляет зам. командира по тылу. Здесь 
		же присутствует зам. командира по вооружению.
		\item вспомогательные пункты управления (разворачиваются на изолированных или удалённых
		участках)
		\item центр боевого управления авиацией (для управления боевыми действиями 
		истребительной авиации и для обеспечения безопасности в зоне ответственности)
		\item пункты управления авиацией (могут быть как на удалённых участках, так и могут 
		совмещаться с основным командным пунктом)
	\end{enumerate}
	Средства управления включают в себя:
	\begin{enumerate}
		\item автоматизированные системы управления
		\item систему связи и радиотехнического обеспечения
	\end{enumerate}
	Все средства управления должны помогать осуществлять как централизованное, так и
децентрализованное управление.

	Виды связей в корпусе или дивизии ПВО:
	\begin{enumerate}
		\item радиорелейная
		\item тропосферная
		\item спутниковая
		\item проводная
		\item подвижная и сигнальная
	\end{enumerate}
	Для передачи или приёма различных видов информации, доставки документов организуется:
	\begin{enumerate}
		\item телефонная связь
		\item видеотелефонная связь
		\item телеграфная связь
		\item факсимильная связь
		\item фельдьегерско-почтовая
	\end{enumerate}
	\newpage
	\centerline{11 марта 2010г.}
	\section{\bf АСУ}
	АСУ корпуса или дивизии ПВО представляют собой совокупность взаимосвязанных комплексов
средств автоматизации командных пунктов корпуса или дивизии, автоматизированных средств
управления подразделениями, частями и соединениями, автоматизированных систем управления
боевыми средствами соединений, частей и подразделений и комплексов средств автоматизации
специального назначения, применяемых по единому плану и замыслу при решении совместных задач.

	АСУ соединениями, частями и подразделениями обеспечивает 
	\begin{enumerate}
		\item сбор, обработку, хранение и выдачу данных обстановки на средства отображения и 
		документирования
		\item приём сигналов боевого управления и доведения их до подчинённых
		\item проведение оперативно-тактических расчётов при выработке решения и планировании 
		боевых действий
		\item доведение до подчинённых частей и подразделений боевых задач и получение 
		донесений о боевом состоянии и боевых действиях частей и подразделений
		\item контроль за подготовкой к выполнению задач подчинёнными и управления частями, 
		подразделениями в ходе выполнения этих боевых задач
	\end{enumerate}
	
	АСУ боевыми средствами предназначены для управления боевыми комплексами и средствами 
поражения с целью максимальной реализации их боевых возможностей. Они обеспечивают обнаружение, 
опознавание, целеобнаружение и наведение на цели при выполнении боевых задач.
	
	{\em АСУ специального назначения} предназначены для повышения эффективности управления при 
решении задач по сбору, обработке и доведения до органов управления подчинённых частей и 
подразделений специальной информации (о боевой готовности пунктов управления, их оснащении, 
частотах средств связи и др.)

	Эффективность применения АСУ обеспечивается:
	\begin{enumerate}
		\item постоянной готовностью их к боевой работе
		\item согласованным и комплексным их применением во всех звеньях управления
		\item подготовкой командиров, органов и пунктов управлений к применению средств
		автоматизации
		\item устойчивой работой системы связи
		\item применением средств повышения достоверности получаемой информации
		\item организацией защиты информации, хранящейся и циркулирующей в комплексах средств 
		автоматизации от несанкционированного доступа и технических средств разведки противника
	\end{enumerate}
	\newpage
	\centerline{18 марта 2010 г.}
	Обеспечение АСУ:
	\begin{enumerate}
		\item организационное обеспечение
		\item методическое обеспечение
		\item информационное обеспечение
		\item математическое обеспечение
		\item программное обеспечение
		\item техническое обеспечение
		\item лингвистическое обеспечение
		\item эргономическое обеспечение
		\item правовое обеспечение
	\end{enumerate}
	Классификация задач, решаемых на ЭВМ:
	\begin{enumerate}
		\item по целевому назначению
			\begin{enumerate}
				\item оперативно-тактические
				\item инженерные
				\item учётно-плановые
			\end{enumerate}
		\item по характеру обработки данных
			\begin{enumerate}
				\item информационные
				\item расчётные
			\end{enumerate}
		\item по области практического применения
			\begin{enumerate}
				\item штабные
				\item исследовательские
				\item учебные
			\end{enumerate}
	\end{enumerate}
	Показатели эффективности работы:
	\begin{enumerate}
		\item боевая готовность
		\item ёмкость
		\item пропускная способность
		\item оперативность
		\item качество решения задач
		\item помехоустойчивость
		\item живучесть
	\end{enumerate}
	\newpage
	\centerline{15.04.2010}
	\section{\bf Боевое дежурство}
	\subsection{\bf Приготовление}
	Надо пройти и сдать зачёты. Отвечает за сдачу начальник штаба. Без допуска на боевое 
дежурство нельзя пройти.

	Состав боевого расчёта (как сокращённого, так и полного) указывается ежеегодно в приказе. 
Там же определены обязанности.
	\subsection{\bf Заступление}
	2 стадии подготовки:
	\begin{enumerate}
		\item Предварительная (за сутки). Организует оперативный дежурный. После этого отдых.
		Без допуска от врача на дежурство не допускают.
		\item Непосредственная. Задаются контрольные вопросы.
	\end{enumerate}
	Через ритуальную площадку взвод проходит на боевое дежурство.
	
	По ритуалу имеется приказ, регламентирующий его. В радиотехнических войсках и истребительной
авиации заступают на сутки. В ЗРВ заступают дивизионами. Боевое дежурство проводится в
соответствии с боевой готовностью №2, а в готовность №1 переводится только по определённым
причинам: есть цель, обнаружен нарушитель...

	\section{\bf Обеспечение боевых действий}
	\centerline{29 апреля 2010 г.}
	Обеспечение боевых действий включает в себя:
	\begin{enumerate}
		\item Боевое обеспечение. Включает в себя:
		\begin{enumerate}
			\item Разведку
			\item Инженерное обеспечение
			\item Радиационная, химическая и биологическая защита
			\item Топогеодезическое обеспечение
			\item Маскировка
			\item Непосредственное прикрытие и наземная оборона
		\end{enumerate}		
		\item Техническое обеспечение. Включает в себя:
		\begin{enumerate}
			\item Ядерно-техническое обеспечение
			\item Инженерно-ракетное обеспечение
			\item Ракетно-техническое обеспечение
			\item Инженерно-радиоэлектронное обеспечение
			\item Инженерно-артиллерийское обеспечение
			\item Инженерно-техническое обеспечение
			\item Химико-техническое обеспечение
			\item Техническое обеспечение связей и АСУ
			\item Авто-техническое обеспечение
			\item Техническое обеспечение по службе тыла
			\item Метрологическое обеспечение
		\end{enumerate}
		\item Тыловое обеспечение. Включает в себя:
		\begin{enumerate}
			\item Материальное обеспечение
			\item Медицинское обеспечение
			\item Ветеринарное обеспечение
			\item Транспортное обеспечение
			\item Квартирно-эксплуатационное обеспечение
			\item Торгово-бытовое обеспечение
			\item Финансовое обеспечение
		\end{enumerate}
		\item Морально-психологическое обеспечение
	\end{enumerate}
	\newpage
	\section{\bf Морально-психологическое обеспечение}
	\centerline{6 марта 2010 г.}
	Целью морально-психологического обеспечения является поддержание боевого духа.
	Основные задачи морально-психологического обеспечения являются:
	\begin{enumerate}
	\item Планирование войск и разъяснение государственной политики в области обороны и безопасности российской безопасности
	\item Доведение решений военно-политического руководства страны (указов, доктрин и т.п.) до личного состава
	\item Формирование у личного состава готовности и способности успешно выполнять поставленные задачи
	\item Достижение морально-психологического превосходства над противником
	\item Поддержание правопорядка и воинской дисциплины в войсках
	\item Снижение психогенных потерь
	\item Поддержание и своевременное восстановление духовных и физических сил военнослужащих
	\item Создание благоприятной морально-психологической обстановки в районе ведения боевых действий
	\item Защита войск и населения от информационно-психологического воздействия противника
	\item Социальная защита военно-служащих и членов их семей
	\item Культурно-художественное обслуживание войск
	\item Обеспечение войск техническими средствами воспитания
\end{enumerate}

	К основным видам относят:
\begin{enumerate}
	\item Информационно-воспитательная работа
	\item Психологическая работа
	\item Военно-специальная работа
	\item Культурно-досуговая работа
	\item Защита войск от информационно-психологического воздействия противника
	\item Обеспечение войск техническими средствами воспитания
\end{enumerate}

	Принципы морально-психологического обеспечения:
\begin{enumerate}
	\item Решительность
	\item Напряжённость
	\item Скоротечность
	\item Высокая манёвренность
	\item Непрерывного наращивания
\end{enumerate}
	
	Основными формами являются:
\begin{enumerate}
\item Психологическая помощь
\item Психологическое сопровождение
\item Социально-психологическая реабилитация
\end{enumerate}
\end{document}